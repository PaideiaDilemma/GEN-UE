%%%%%%%%%%%%%%%%%%%%%%%%%%%%%%%%%%%%%%%%%%%%%%%%%%%%%%%%%%%%%%%%%%
%%								%%
%%	mmm  mmmmmm mm   m        m    m mmmmmm mmm     mmmm	%%
%%    m"   " #      #"m  #        #    # #        #    m"  "m	%%
%%    #   mm #mmmmm # #m #        #    # #mmmmm   #    #  m #	%%
%%    #    # #      #  # #  """   #    # #        #    #    #	%%
%%     "mmm" #mmmmm #   ##        "mmmm" #mmmmm mm#mm   #mm#	%%
%%								%%
%%							      	%%
%%                                                              %%
%%                                                              %%
%%      Grundlagen der Elektrischen Netzwerke, UE               %%
%%      Gruppe 5, Team F                                        %%
%%      Authors: Severin Wolf, Maximilian Seidler.              %%
%%%%%%%%%%%%%%%%%%%%%%%%%%%%%%%%%%%%%%%%%%%%%%%%%%%%%%%%%%%%%%%%%%
\documentclass[a4paper]{article}
\usepackage{amsmath}
\usepackage[utf8]{inputenc}
\usepackage[T1]{fontenc}
\usepackage[english]{babel}
\usepackage{geometry}
\usepackage{graphicx}
\usepackage{tikz}
\usepackage{listings}
\usepackage{trfsigns}
\usepackage{polynom}
\geometry{a4paper,left=3cm,right=2cm, top=2cm, bottom=2cm} 
\usepackage[EFvoltages, european, straightvoltages]{circuitikz}

%tikz
\ctikzset{resistor = european}
\usetikzlibrary{decorations.pathreplacing}

%no paragraph indent
\setlength{\parindent}{0pt}

%for math, that does not fit
\renewcommand*{\arraystretch}{1.3}
\newcommand\scalemath[2]{\scalebox{#1}{\mbox{\ensuremath{\displaystyle #2}}}}

\newcommand\blfootnote[1]{%
	\begingroup
	\renewcommand\thefootnote{}\footnote{#1}%
	\addtocounter{footnote}{-1}%
	\endgroup
}

\polyset{%
  style=C,
  delims={\big(}{\big)},
  div=:
}

\begin{document}
\pagestyle{empty} \enlargethispage*{25cm}\samepage{
\vspace*{-3cm}
\begin{center}
\begin{minipage}[!h]{16cm}
\hspace*{0.2cm}
\includegraphics[width=3.3cm]{./Figures/igte_logo}
\begin{tabular}{p{8cm}}
\vspace{0.2cm}
\centering{
\Large Institute of Fundamentals and Theory in
 Electrical Engineering\\
Graz University of Technology\\
~\\}
\end{tabular}
\includegraphics[width=3.3cm]{./Figures/TUG_logo}
\end{minipage}
	
		\vspace*{0.5cm}
		\Large
		\textbf{Fundamentals of electrical circuits} \\
		\textbf{10. Homework}\\
		Realization of Multiports
		\vspace*{0.5cm}
		
		\large
		10.06.2021
	\end{center}}
	
	\vspace*{0.2cm}
	
	%%%%%%%%%%%%%%%%%%%%%%%%%%%%%%%%%%%%%%%%%%%%%%%%%%%%%%%%%%%%%%%%%%%%
	%%%%%%%%%%%%%%%%%%%%%%%%%%%%%%%%%%%%%%%%%%%%%%%%%%%%%%%%%%%%%%%%%%%%
	\section*{Assignment 10}

The following impedance parameters are given:
%\vspace{0.2cm}\\
\begin{alignat*}{1}
Z_{11} &= \frac{48 s^3+232s^2+120s}{24s^2+32s+10} \\
Z_{12} &= Z_{21}=\frac{30s}{6s+5} \\
Z_{22} &= \frac{42s^3+16s^2+47s+10}{6s^3+5s^2+6s+5}
\end{alignat*}

%\vspace{0.5cm}
\textbf{1. Analytical calculations:}

\begin{itemize}
	\item Determine the three impedances of a T-Network by using the given impedance parameters. Can one say anything about the multiport's symmetry by looking at the given set of impedance parameters? (\textit{1.5~P})
	\item Find the realization of the multiport which corresponds to the given set of parameters by combining the elements R, L and C. Draw the resulting circuit with all values. (\textit{1.5~P})
	\item Determine the transfer function of the circuit $H(s) = \frac{U_2(s)}{U_1(s)}\Big|_{I_2 = 0}$. (\textit{0.5~P})
\end{itemize}

\textbf{2. Matlab:}

\begin{itemize}
	\item Check the results from the analytical calculations in Matlab by calculating the impedances $Z_{11},Z_{12},Z_{22}$ from the multiport's realization. Do your results match with the given impedance parameters? (\textit{0.25~P}) \\
	\textit{hint: you might have to cancle some zeros / poles... Matlab-command: minreal()}
	\item Implement the calculation of the chain-matrix in Matlab. Use the correct element of the chain-matrix in order to calculate the transfer function. (\textit{0.25~P})
	\item Construct a bode-plot of the transfer function. (\textit{0.25~P})

\end{itemize}

\textbf{3. LTSice}

\begin{itemize}
	\item Simulate the derived circuit in LTSpice. Generate a bode-plot and compare it with the Matlab-result. (\textit{0.75~P})
\end{itemize}

\begin{table}[h]
	\centering
	\renewcommand{\arraystretch}{1.3}
		\begin{tabular}{l||c|c|c|c|c|c}
				& $c_1$& $c_2$& $c_3$& $c_4$& $c_5$& $c_6$ \\ \hline \hline
		$\left[\underline{Z}\right]$ &$ 1 $& $\underline{Z}_{11}$ &$\underline{Z}_{12}$&$\underline{Z}_{21}$&$\underline{Z}_{22}$&$det\left[\underline{Z}\right]$ \\ \hline
	$	\left[\underline{Y}\right]$ &$det\left[\underline{Y}\right]$&$\underline{Y}_{22}$&$-\underline{Y}_{12}$&$-\underline{Y}_{21}$&$\underline{Y}_{11}$&$1$  \\ \hline
	$\left[\underline{A}\right]$ &$\underline{A}_{21} $& $\underline{A}_{11}$ &$det\left[\underline{A}\right]$&$1$&$\underline{A}_{22}$&$\underline{A}_{12}$
		\end{tabular}
	\caption{Parameter Conversion Table}
\end{table}
%\vspace*{1cm}
\blfootnote{Submission deadline: 17 June 2021;  \qquad presentation: team J \\
	 \qquad enjoy your holidays ;)}

\clearpage
\section{Analytical calculations}
\subsection{Impedances of a T-Network}

\begin{figure}[!h] \centering
  \begin{circuitikz}
    \draw(0,0)
    to[short, *-o](-3.5, 0)
    to[short, -o](3.5,0);
    \draw(-3.5,5)
    to[short, o-](-3.4,5)
    to[short, i=$\underline{I_1}$, color=red](-3.2,5)
    to[R, l=\raisebox{1mm}{$\underline{Z_a}$}, -*] (0,5)
    to[R, l=\raisebox{1mm}{$\underline{Z_c}$}] (3.2,5)
    to[short, i<=$\underline{I_2}$, color=red](3.4,5)
    to[short,-o](3.5,5);
    \draw(0,0)
    to[short](0,1)
    to[R, l=$\underline{Z_b}$](0,5);
    \draw[-{Latex[length=2mm]}, color=blue] (-3.5, 4.5) -- (-3.5, 0.5)
    node[pos=0.5, left] {$\underline{U_{1}}$};
    \draw[-{Latex[length=2mm]}, color=blue] (3.5, 4.5) -- (3.5, 0.5)
    node[pos=0.5, right] {$\underline{U_{2}}$};
  \end{circuitikz}	
  \caption{A typical T-Network}
  \label{fig:tnet}
\end{figure}
Using the given elements of the impedance matrix $\begin{bmatrix} \underline{Z} \end{bmatrix}$, we
want to calculate the elements $\underline{Z}_{a}, \underline{Z}_{b}, \underline{Z}_{c}$ of a 
t-network like the one in Figure \ref{fig:tnet}. This can be done by using the scheme to calculate 
the impedance matrix of a t-network in the first place. 
\[
  \begin{bmatrix}
    \underline{Z}
  \end{bmatrix} = 
  \begin{bmatrix}
    \underline{Z}_{a} + \underline{Z}_{b} & \underline{Z}_{b} \\
    \underline{Z}_{b} & \underline{Z}_{b} + \underline{Z}_{c}
  \end{bmatrix}
.\] 
Element $\underline{Z}_{b}$ directly corresponds to the given value for $\underline{Z}_{12} =
\underline{Z}_{21}$. The Elements $\underline{Z}_{a}$ and $\underline{Z}_{b}$ have to be calculated
by substraction.
\subsubsection*{Calculate $\underline{Z}_{a}$:}
\[
  \underline{Z}_{11} = \underline{Z}_{a} + \underline{Z}_{b} \quad \rightarrow \quad 
  \underline{Z}_{a} = \underline{Z}_{11}-\underline{Z}_{b}
.\] 
Because $\underline{Z}_{11}$ and $\underline{Z}_{b}$ are fractions of polynomials, lets see, if 
they have common factors in theire denomitators. That would make a substraction much easier.
\[
  Z_{11} = \frac{48 s^3+232s^2+120s}{24s^2+32s+10}, \quad
  Z_{b} = \frac{30s}{6s+5} 
.\] 
Factorize the denomitator of $\underline{Z}_{11}$:
\[
  24s^2 + 32s + 10 = 2(12s^2 + 16s + 5) = 2(2s + 1)(6s + 5)
.\] 
Since there is a common denomitator, lets extend the polynomial fraction of $\underline{Z}_{b}$ with
$2(2s+1)$ and do the substraction.
\[
  \underline{Z}_{a} = \frac{48 s^3+232s^2+120s- 2(2s+1)(30s)}{2(2s+1)(6s+5)} = 
  \frac{48 s^3+232s^2+120s-120s^2-60s}{2(2s+1)(6s+5)}
.\] 
\[
  \underline{Z}_{a} = \frac{48s^3 + 112s^2 + 60s}{24s^2 + 32s +10}
.\] 
\subsubsection*{Calculate $\underline{Z}_{c}$:}
\[
  \underline{Z}_{22} = \underline{Z}_{b} + \underline{Z}_{c} \quad \rightarrow \quad 
  \underline{Z}_{c} = \underline{Z}_{22}-\underline{Z}_{b}
.\] 
with
\[
  \underline{Z}_{22} = \frac{42s^3+16s^2+47s+10}{6s^3+5s^2+6s+5}
  \underline{Z}_{b} = \frac{30s}{6s+5} 
.\] 
Once again, lets find a common denomitator of those polynomial fractions, by factorizing the
denomitator of $\underline{Z}_{22}$:
\[
  6s^3 + 5s + 6s + 5 = (s^2 + 1)(6s + 5)
.\] 
Now extend the fraction of $\underline{Z}_{b}$ with $s^2 + 1$ and subtract.
\[
  \underline{Z}_{c} = \frac{42s^3+16s^2+47s+10-(s^2+1)(30s)}{(s^2+1)(6s+5)} =
  \frac{42s^3+16s^2+47s+10- 30s^3 - 30s)}{(s^2+1)(6s+5)}
.\] 
\[
  \underline{Z}_{c} = \frac{12s^3 + 16s^2 + 17s +10}{6s^3 + 5s^2 + 6s+ 5}
.\] 

\clearpage
\subsubsection*{Symmetry of the multiport}
This network is not symetrical, since its input and output impedances are not the same $
\underline{Z}_{22} \ne \underline{Z}_{22}$. However, like every t-network, it is reciprocal, since $\underline{Z}_{12} =
\underline{Z}_{21}$.
\subsection{Find the realization of the multiport}
Now, that we have the impedances of our network, we want to convert them into real values.
\subsubsection*{$R, L$ and  $C$ values for $\underline{Z}_{a}$:}
\[
  \underline{Z}_{a} = \frac{48s^3 + 112s^2 + 60s}{24s^2 + 32s +10}
.\] 
First, we will perform a polynomial division, to simplify the term and pull values out of the
fraction.
\[
  \polylongdiv[vars=s]{48s^3 + 112s^2 + 60s}{24s^2 + 32s +10} 
\] 
We can simplify it even further:
\[
  \underline{Z}_{a} = 2s + 2 + \frac{-4(6s + 5)}{2(2s+1)(6s+5)} = 2s + 2 - \frac{2}{2s+1}
.\] 
To get rid of the minus;
\[
  2s + 2 - \frac{2}{2s+1} = 2s-\frac{4s+2-2}{2s+1} = 2s-\frac{4}{2 + \frac{1}{s}}
.\] 
Finally, we can say something about the elements, that could produce such an impedance:
\[
  2s + \frac{4}{2+\frac{1}{s}} \overset{\wedge}{=} L_{1}s + \underline{Z}_{a}^{(2)}
.\] 
\[
  \rightarrow \quad L_{1} = 2H
.\] 
$\underline{Z}_{a}^{(2)}$ is probaly a parallel circuit. Thats, why we invert them, to be able
to look at attmitances (which can be summed up).
\[
  \underline{Y}_{a}^{(2)} = \frac{1}{2} + \frac{1}{4s} \overset{\wedge}{=} \frac{1}{R_1} +
  \frac{1}{sL_{2}}
.\] 
\[
  \rightarrow \quad R_{1} = 2\Omega, \quad L_{2} = 4H
.\] 
\begin{figure}[ht] \centering 
  \begin{circuitikz}
    \draw (0,0)
    to[L, l=$2H$, o-*] (3,0)
    to[short] (3,1)
    to[R, l=$2\Omega$] (6,1)
    to[short] (6,0)
    (3,0)
    to[short] (3,-1)
    to[L, l=$4H$] (6,-1)
    to[short] (6,0)
    to[short, *-o] (7,0);
  \end{circuitikz}
  \caption{Circuit derived from the impedance $\underline{Z}_{a}$}
  \label{fig:z_a}
\end{figure}
\clearpage
\subsubsection*{$R, L$ and  $C$ values for $\underline{Z}_{b}$:}
\[
  \underline{Z}_{b} = \frac{30s}{6s+5} = \frac{30}{6+\frac{5}{s}}
.\] 
Convert it to an attmitance:
\[
  \underline{Y}_{b} = \frac{1}{5} + \frac{1}{6s} \overset{\wedge}{=} \frac{1}{R_{2}} +
  \frac{1}{sL_{3}}
.\] 
\[
  \rightarrow \quad R_{2} = 5\Omega, \quad L_{3} = 6H
.\] 
\begin{figure}[ht] \centering 
  \begin{circuitikz}
    \draw (0,0)
    to[short, o-*] (1,0)
    to[short] (1,1)
    to[R, l=$5\Omega$] (4,1)
    to[short] (4,0)
    (1,0)
    to[short] (1,-1)
    to[L, l=$6H$] (4,-1)
    to[short] (4,0)
    to[short, *-o] (5,0);
  \end{circuitikz}
  \caption{Circuit derived from the impedance $\underline{Z}_{b}$}
  \label{fig:z_b}
\end{figure}
\subsubsection*{$R, L$ and  $C$ values for $\underline{Z}_{c}$:}
\[
  \underline{Z}_{c} = \frac{12s^3 + 16s^2 + 17s +10}{6s^3 + 5s^2 + 6s+ 5}
.\] 
Polynomial division:
\[
  \polylongdiv[vars=s]{12s^3 + 16s^2 + 17s +10}{6s^3 + 5s^2 + 6s+ 5}
.\] 
Further simplification:
\[
  \underline{Z}_{c} = 2+\frac{s(6s+5)}{(s^2+1)(6s+5)}
  2+ \frac{s(6s+5)}{s(s+\frac{1}{s})(6s+5)} = 2 + \frac{1}{s + \frac{1}{s}}
.\] 
\[
  \rightarrow \quad R_{3} = 2 \Omega 
.\] 
Invert the fraction to a admittance:
\[
  \underline{Y}_{c}^{(2)} = s + \frac{1}{s} \overset{\wedge}{=} sC_{1} + \frac{1}{sL_{4}}
.\] 
\[
  \rightarrow  \quad C_{1} = 1F, \quad L_{4} = 1H
.\] 
\begin{figure}[ht] \centering 
  \begin{circuitikz}
    \draw (0,0)
    to[R, l=$2\Omega$, o-*] (3,0)
    to[short] (3,1)
    to[C, l=$1F$] (6,1)
    to[short] (6,0)
    (3,0)
    to[short] (3,-1)
    to[L, l=$1H$] (6,-1)
    to[short] (6,0)
    to[short, *-o] (7,0);
  \end{circuitikz}
  \caption{Circuit derived from the impedance $\underline{Z}_{c}$}
  \label{fig:z_c}
\end{figure}

\clearpage
\subsubsection*{The realized circuit:}
\begin{figure}[ht] \centering 
  \begin{circuitikz}[scale=0.75, transform shape]
    \draw (0,5)
    to[L, l=$2H$, o-*] (3,5)
    to[short] (3,6)
    to[R, l=$2\Omega$] (6,6)
    to[short] (6,5)
    (3,5)
    to[short] (3,4)
    to[L, l=$4H$] (6,4)
    to[short] (6,5)
    to[short, *-] (8,5);
    \draw (8,5)
    to[short, -*] (8,4)
    to[short] (9,4)
    to[R, l=$5\Omega$] (9,1)
    to[short] (8,1)
    (8,4)
    to[short] (7,4)
    to[L, l=$6H$] (7,1)
    to[short] (8,1)
    to[short, *-*] (8,0);
    \draw (8,5)
    to[R, l=$2\Omega$, *-*] (11,5)
    to[short] (11,6)
    to[C, l=$1F$] (14,6)
    to[short] (14,5)
    (11,5)
    to[short] (11,4)
    to[L, l=$1H$] (14,4)
    to[short] (14,5)
    to[short, *-o] (15,5);
    \draw (0,0)
    to[short, o-o] (15,0);
    \draw[-{Latex[length=2mm]}, color=blue] (0, 4.5) -- (0, 0.5)
    node[pos=0.5, left] {$\underline{U_{1}}$};
    \draw[-{Latex[length=2mm]}, color=blue] (15, 4.5) -- (15, 0.5)
    node[pos=0.5, right] {$\underline{U_{2}}$};
  \end{circuitikz}
  \caption{Derived Circuit}
  \label{fig:all}
\end{figure}









\end{document}
